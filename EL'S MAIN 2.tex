\documentclass{article}

% Language setting
% Replace `english' with e.g. `spanish' to change the document language
\usepackage[english]{babel}

% Set page size and margins
% Replace `letterpaper' with `a4paper' for UK/EU standard size
\usepackage[letterpaper,top=2cm,bottom=2cm,left=3cm,right=3cm,marginparwidth=1.75cm]{geometry}

% Useful packages
\usepackage{amsmath}
\usepackage{graphicx}
\usepackage[colorlinks=true, allcolors=blue]{hyperref}

\title{Assignment 3 (Diode Signal Detector on ESP32)}
\author{Elgiva Nhyiraba Amanquanor Annan, 4090324}

\begin{document}
\maketitle


\section{Introduction}

This report outlines the third lab session, focusing on the construction and testing of a diode-based signal detection circuit with an ESP32 microcontroller. The objective was to safely detect high-voltage signals using a diode for protection.

\section{ GitHub Repository Screenshot}

\begin{figure}[h]
    \centering
    \includegraphics[width=1.0 \textwidth]{Diode.png}
    \caption{ GitHub Repository with Project Code}
    \label{fig:image}
\end{figure}

\section{What is a Diode and How it Functions}
A diode is a semiconductor component that permits current to flow in only one direction. It features两位 terminals: the anode and the cathode. Current flows when the anode's voltage exceeds the cathode's (forward-biased), but is blocked when the polarity reverses (reverse-biased). This one-way property makes diodes valuable for safeguarding circuits against reverse voltage and ensuring unidirectional signal flow, which is essential in this experiment for detecting high voltage post-rectification.
\section{Project Description}
The project focused on using the ESP32 to detect whether a high voltage signal was present after a diode in the circuit.

\subsection{Steps Followed}

 \item  1. Built a basic rectifier circuit using a diode to clip AC signals.
 \item  2. Connected the output of the diode to GPIO pin 25 on the ESP32.
 \item  3. Wrote Arduino code to read the input and print messages based on voltage level.
 \item  4. Uploaded and monitored the serial monitor output to confirm detection.

\subsection{Arduino Code Used}

\item // Diode Signal Detector on ESP32
\item const int signalPin = 25; // Pin connected after the diode
\item void setup()
  \{
\item Serial.begin(115200);
\item pinMode(signalPin, INPUT);
 \}
\item void loop()
  \{
\item int signalState = digitalRead(signalPin);
\item if (signalState == HIGH)
 \{
\item Serial.println("High voltage detected!");
\}
\item else
  \{
\item Serial.println("No high voltage detected (safe).");
\}
\item delay(500);
\}


\section{Observations and Results}
The code successfully detected voltage levels. When a signal was present after the diode,the serial monitor displayed “High voltage detected!”. When no signal was present, the message was “No high voltage detected (safe).”
This behavior validated that the diode effectively allowed only forward-biased signals and protected the ESP32 from direct exposure to higher voltages.

\newpage

\section{ Proteus Simulations}
\subsection{ Experiment 1 and 2 Screenshot}
\begin{figure}[h]
    \centering
    \includegraphics[width=1.0 \textwidth]{Proteus.png}
    \caption{Proteus Simulation - Experiment 1}
    \label{fig:image}
\end{figure}

\section{Conclusion}
The lab effectively showcased voltage detection following a diode using an ESP32. The diode offered critical protection, and the ESP32 consistently identified the presence or absence of a signal.
\end{document}
